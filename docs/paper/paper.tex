\documentclass{article}



\usepackage{arxiv}

\usepackage[utf8]{inputenc} % allow utf-8 input
\usepackage[T1]{fontenc}    % use 8-bit T1 fonts
\usepackage{hyperref}       % hyperlinks
\usepackage{url}            % simple URL typesetting
\usepackage{booktabs}       % professional-quality tables
\usepackage{amsfonts}       % blackboard math symbols
\usepackage{nicefrac}       % compact symbols for 1/2, etc.
\usepackage{microtype}      % microtypography
\usepackage{lipsum}		% Can be removed after putting your text content
\usepackage{graphicx}
\usepackage{natbib}
\usepackage{doi}



\title{ZPY: Open Source Synthetic Data for Computer Vision}

% \date{June 18, 2021}

\author{
	{\hspace{1mm}Hugo Ponte}\thanks{correspondence author} \\ 
	Zumo Labs \\
	\texttt{hugo@zumolabs.ai} \\
	%% examples of more authors
	\And
	{\hspace{1mm}Norman Ponte} \\
	Zumo Labs \\
	\texttt{norman@zumolabs.ai} \\
	\And
	{\hspace{1mm}Sammie Crowder} \\
	Zumo Labs \\
	\texttt{sammie@zumolabs.ai} \\
	\And
	{\hspace{1mm}Kory Stiger} \\
	Zumo Labs \\
	\texttt{kory@zumolabs.ai} \\
	\And
	{\hspace{1mm}Steven Pecht} \\
	Zumo Labs \\
	\texttt{steven@zumolabs.ai} \\
	\And
	{\hspace{1mm}Elena Ponte} \\
	Zumo Labs \\
	\texttt{elena@zumolabs.ai} \\
}

% Uncomment to remove the date
\date{}

% Uncomment to override  the `A preprint' in the header
\renewcommand{\headeright}{Technical Report}
\renewcommand{\undertitle}{Technical Report}
%\renewcommand{\shorttitle}{\textit{arXiv} Template}

%%% Add PDF metadata to help others organize their library
%%% Once the PDF is generated, you can check the metadata with
%%% $ pdfinfo template.pdf
\hypersetup{
pdftitle={ZPY: Open Source Synthetic Data for Computer Vision},
pdfsubject={cs.CV},
pdfauthor={Hugo Ponte, Norman Ponte, Sammie Crowder, Kory Stiger, Steven Pecht, Elena Ponte},
pdfkeywords={Computer Vision, Synthetic Data, Machine Learning, Open Source, Python, Blender},
}

\begin{document}
\maketitle

\begin{abstract}
Synthetic data presents a unique solution to the huge data requirements of computer vision with deep learning.
In this work, we present zpy, an open source framework for creating synthetic data. Built on top of Blender,
and designed with modularity and readability in mind.
\end{abstract}

\keywords{Computer Vision \and Synthetic Data \and Machine Learning \and Open Source \and Python \and Blender}

\section{Introduction}
\label{sec:introduction}
\lipsum[2]

\section{Background}
\label{sec:background}
\lipsum[2]

\subsection{3D}
\lipsum[5]

\subsection{Deep Learning}
\lipsum[5]

\subsection{Synthetic Data}
\lipsum[5]

\section{Motivation}
\label{sec:motivation}

\subsection{Democratization of Data}
\lipsum[2]

\subsection{Fairness and Bias}
\lipsum[2]

\subsection{Software 3.0}
\lipsum[2]

\section{Project Features}
\label{sec:projectfeatures}

\subsection{Blender Addon}
\lipsum[2]

\subsection{Cloud Backend}
\lipsum[2]

\subsection{User Interfaces}
\lipsum[4] See Section \ref{sec:background}.

\section{Design}
\label{sec:design}

\section{Workflow}
\label{sec:workflow}

\section{Case Study}
\label{sec:casestudy}

\section{Conclusion}
\label{sec:casestudy}

\subsection{Python First}
\lipsum[2]

\subsection{Headings: second level}
\lipsum[5]
\begin{equation}
	\xi _{ij}(t)=P(x_{t}=i,x_{t+1}=j|y,v,w;\theta)= {\frac {\alpha _{i}(t)a^{w_t}_{ij}\beta _{j}(t+1)b^{v_{t+1}}_{j}(y_{t+1})}{\sum _{i=1}^{N} \sum _{j=1}^{N} \alpha _{i}(t)a^{w_t}_{ij}\beta _{j}(t+1)b^{v_{t+1}}_{j}(y_{t+1})}}
\end{equation}

\subsubsection{Headings: third level}
\lipsum[6]

\paragraph{Paragraph}
\lipsum[7]



\section{Examples of citations, figures, tables, references}
\label{sec:others}

\subsection{Citations}
Citations use \verb+natbib+. The documentation may be found at
\begin{center}
	\url{http://mirrors.ctan.org/macros/latex/contrib/natbib/natnotes.pdf}
\end{center}

Here is an example usage of the two main commands (\verb+citet+ and \verb+citep+): Some people thought a thing \citep{kour2014real, hadash2018estimate} but other people thought something else \citep{kour2014fast}. Many people have speculated that if we knew exactly why \citet{kour2014fast} thought this\dots

\subsection{Figures}
\lipsum[10]
See Figure \ref{fig:fig1}. Here is how you add footnotes. \footnote{Sample of the first footnote.}
\lipsum[11]

\begin{figure}
	\centering
	\fbox{\rule[-.5cm]{4cm}{4cm} \rule[-.5cm]{4cm}{0cm}}
	\caption{Sample figure caption.}
	\label{fig:fig1}
\end{figure}

\subsection{Tables}
See awesome Table~\ref{tab:table}.

The documentation for \verb+booktabs+ (`Publication quality tables in LaTeX') is available from:
\begin{center}
	\url{https://www.ctan.org/pkg/booktabs}
\end{center}


\begin{table}
	\caption{Sample table title}
	\centering
	\begin{tabular}{lll}
		\toprule
		\multicolumn{2}{c}{Part}                   \\
		\cmidrule(r){1-2}
		Name     & Description     & Size ($\mu$m) \\
		\midrule
		Dendrite & Input terminal  & $\sim$100     \\
		Axon     & Output terminal & $\sim$10      \\
		Soma     & Cell body       & up to $10^6$  \\
		\bottomrule
	\end{tabular}
	\label{tab:table}
\end{table}

\subsection{Lists}
\begin{itemize}
	\item Lorem ipsum dolor sit amet
	\item consectetur adipiscing elit.
	\item Aliquam dignissim blandit est, in dictum tortor gravida eget. In ac rutrum magna.
\end{itemize}


\bibliographystyle{unsrtnat}
\bibliography{references}

\end{document}
